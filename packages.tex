\usepackage{fullpage}
%\usepackage{algorithm,algorithmic}
\usepackage[ruled]{algorithm2e}
\usepackage{algpseudocode}
%\usepackage{cite}
\usepackage[utf8]{inputenc}
\usepackage[english]{babel}
\usepackage{amsthm}
\newtheorem{theorem}{Theorem}[section]
\newtheorem{corollary}{Corollary}[theorem]
\newtheorem{remark}{Remark}[section] 
\usepackage{color}
\usepackage{amsmath}
\usepackage{amsfonts}
\usepackage{mathtools}
\usepackage[titletoc,title]{appendix}
%======
\usepackage{float}
%\usepackage{wrapfig,blindtext}
\usepackage{graphicx} % to include images
\usepackage{epstopdf} % to convert eps to pdf
\usepackage{caption}[font=small,skip=-2pt]
\usepackage{subcaption}[font=small,skip=-2pt]
\usepackage{pdfpages}
\usepackage{comment}

\usepackage{hyperref}
\hypersetup{
    colorlinks=true,
    linkcolor=blue,
    filecolor=magenta,      
    urlcolor=cyan,
}
\urlstyle{same}
%\setlength{\belowcaptionskip}{-5pt}
%\usepackage{sfmath} 

\usepackage{tikz}
\usetikzlibrary{positioning, fit, arrows.meta, shapes}
\newcommand{\empt}[2]{$#1^{\langle #2 \rangle}$}
\usetikzlibrary{
  arrows.meta, % for Straight Barb arrow tip
  fit, % to fit the group box around the central neurons
  positioning, % for relative positioning of the neurons
}
\tikzset{
  neuron/.style={ % style for each neuron
    circle,draw,thick, % drawn as a thick circle
    inner sep=0pt, % no built-in padding between the text and the circle shape
    minimum size=3.5em, % make each neuron the same size regardless of the text inside
    node distance=1ex and 2em, % spacing between neurons (y and x)
  },
  group/.style={ % style for the groups of neurons
    rectangle,draw,thick, % drawn as a thick rectangle
    inner sep=0pt, % no padding between the node contents and the rectangle shape
  },
  io/.style={ % style for the inputs/outputs
    neuron, % inherit the neuron style
    fill=gray!15, % add a fill color
  },
  conn/.style={ % style for the connections
    -{Straight Barb[angle=60:2pt 3]}, % simple barbed arrow tip
    thick, % draw in a thick weight to match other drawing elements
  },
}
\numberwithin{equation}{section}
