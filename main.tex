%\documentclass{article}
\documentclass[3p,computermodern,10pt]{elsarticle}
\usepackage{fullpage}
%\usepackage{algorithm,algorithmic}
\usepackage[ruled]{algorithm2e}
\usepackage{algpseudocode}
%\usepackage{cite}
\usepackage[utf8]{inputenc}
\usepackage[english]{babel}
\usepackage{amsthm}
\newtheorem{theorem}{Theorem}[section]
\newtheorem{corollary}{Corollary}[theorem]
\newtheorem{remark}{Remark}[section] 
\usepackage{color}
\usepackage{amsmath}
\usepackage{amsfonts}
\usepackage{mathtools}
\usepackage[titletoc,title]{appendix}
%======
\usepackage{float}
%\usepackage{wrapfig,blindtext}
\usepackage{graphicx} % to include images
\usepackage{epstopdf} % to convert eps to pdf
\usepackage{caption}[font=small,skip=-2pt]
\usepackage{subcaption}[font=small,skip=-2pt]
\usepackage{pdfpages}
\usepackage{comment}

\usepackage{hyperref}
\hypersetup{
    colorlinks=true,
    linkcolor=blue,
    filecolor=magenta,      
    urlcolor=cyan,
}
\urlstyle{same}
%\setlength{\belowcaptionskip}{-5pt}
%\usepackage{sfmath} 

\usepackage{tikz}
\usetikzlibrary{positioning, fit, arrows.meta, shapes}
\newcommand{\empt}[2]{$#1^{\langle #2 \rangle}$}
\usetikzlibrary{
  arrows.meta, % for Straight Barb arrow tip
  fit, % to fit the group box around the central neurons
  positioning, % for relative positioning of the neurons
}
\tikzset{
  neuron/.style={ % style for each neuron
    circle,draw,thick, % drawn as a thick circle
    inner sep=0pt, % no built-in padding between the text and the circle shape
    minimum size=3.5em, % make each neuron the same size regardless of the text inside
    node distance=1ex and 2em, % spacing between neurons (y and x)
  },
  group/.style={ % style for the groups of neurons
    rectangle,draw,thick, % drawn as a thick rectangle
    inner sep=0pt, % no padding between the node contents and the rectangle shape
  },
  io/.style={ % style for the inputs/outputs
    neuron, % inherit the neuron style
    fill=gray!15, % add a fill color
  },
  conn/.style={ % style for the connections
    -{Straight Barb[angle=60:2pt 3]}, % simple barbed arrow tip
    thick, % draw in a thick weight to match other drawing elements
  },
}
\numberwithin{equation}{section}

\newcommand{\TODO}[1]{{\color{red}{to do: #1}}}
\newcommand{\timeDummy}{\tau}
\newcommand{\spatialAcronym}{SDR}
\newcommand{\spaceTimeAcronym}{STDR}
\newcommand{\parametricSpaceTimeAcronym}{PSTDR}
\newcommand{\MLSubspaceNameLowercase}{deep subspaces}
%\newcommand{\span}[1]{\text{span}{#1}}
\newcommand{\PSTWeightingMatrix}{\boldsymbol W}
\newcommand{\PSTInnerProduct}[2]{\left( #1, #2 \right)_{*}}
\newcommand{\defeq}{\vcentcolon=}
\newcommand{\RRStar}[2]{\mathbb{V}_{#1}(\RR{#2})}
\newcommand{\Range}[1]{\text{Range}(#1)}
\newcommand{\RomDim}{K}
\newcommand{\NVars}{N_v}
\newcommand{\ParametricSpace}{\mathcal{P}}
\newcommand{\TimeDomain}{[0,T]}
\newcommand{\ParamDomain}{\mathcal{D}}
\newcommand{\paramDomain}{\mathcal{D}}
\newcommand{\numParams}{N_{\mu}}
\newcommand{\numSpaceDims}{N_x}
\newcommand{\SpatialDomain}{\Omega}
\newcommand{\PhysicalDomain}{\Omega}
\newcommand{\PDEStateDomain}{\PhysicalDomain \times \TimeDomain \times \paramDomain}
\newcommand{\PDEStateCodomain}{\RR{\NVars}}
\newcommand{\PDEFlux}{\mathcal{F}}
\newcommand{\NSpace}{N_s}
\newcommand{\NTime}{\numTimeSteps}
\newcommand{\NST}{N_{st}}
\newcommand{\x}{\boldsymbol x}
\newcommand{\xd}{\mathbf{x}}
\newcommand{\xDummy}{\boldsymbol y}
\newcommand{\xs}{x}
\newcommand{\ys}{y}
\newcommand{\zs}{z}
\newcommand{\NeuralNetwork}{\mathcal{NN}}
\newcommand{\TimeSpace}{\mathcal{T}}
\newcommand{\STDim}{K_{\text{st}}}
\newcommand{\PSTDim}{K_{\text{pst}}}

\newcommand{\SemiDiscreteSpaceTimeTrialSpace}{\mathcal{ST}}

\newcommand{\ParametricSpaceTimeTrialSpace}{\mathcal{PXT}}

\newcommand{\SemiDiscreteParametricSpaceTimeTrialSpace}{\mathcal{PST}}

\newcommand{\ParametricSpaceTimeBasisVec}{\boldsymbol \psi}
\newcommand{\ParametricSpaceTimeBasisMat}{\boldsymbol \Psi}

\newcommand{\SpaceTimeBasisVec}{\boldsymbol \pi}
\newcommand{\SemiDiscreteSpaceTimeBasisMat}{\boldsymbol \Pi}
\newcommand{\OnesFunction}{\mathcal{O}}
\newcommand{\SemiDiscreteParametricSpaceTimeBasisMat}{\boldsymbol \Phi}
\newcommand{\SemiDiscreteParametricSpaceTimeBasisVec}{\boldsymbol \phi}

\newcommand{\TrainingData}{\mathcal{S}}
\newcommand{\NTrain}{N_{\text{train}}}
\newcommand{\SemiDiscreteStateIC}{\SemiDiscreteState_0(\params)}
\newcommand{\SemiDiscreteStateParametricSpaceTimeRef}{\SemiDiscreteState_{\text{ref}}}

\newcommand{\RRSym}[1]{\mathbb{R}_{\text{s}}^{#1}}
\newcommand{\RR}[1]{\mathbb{R}^{#1}}
\newcommand{\FomDim}{N}
\newcommand{\ApproxPDEState}{\tilde{\PDEState}}

\newcommand{\PDEStateParametricSpaceTimeTrialSpace}{\textit{PXT}}

\newcommand{\PDEStateSpaceTimeTrialSpace}{\textit{XT}}

\newcommand{\PDEStateSpatialTrialSpace}{\textit{X}}
\newcommand{\SpatialTrialSpace}{\mathcal{X}}
\newcommand{\SpaceTimeTrialSpace}{\mathcal{XT}}

\newcommand{\PDEState}{\mathsf{u}}
\newcommand{\PDEStateRef}{\mathsf{u}_{\text{ref}}}

%\newcommand{\ApproxPDEState}{\tilde{\boldsymbol u}}
\newcommand{\NNWeights}{\boldsymbol \theta}
\newcommand{\NWeights}{N_{\NNWeights}}
\newcommand{\weightsInner}{\boldsymbol \theta^*}
\newcommand{\bz}{\boldsymbol 0}
\newcommand{\params}{\boldsymbol \mu}
\newcommand{\paramsDummy}{\boldsymbol \nu}
\newcommand{\PDEStateArgs}{\x, t,\params}
\newcommand{\weights}{\boldsymbol \theta}
\newcommand{\Basis}{\boldsymbol \Phi}
\newcommand{\SpatialBasisVec}{\boldsymbol \phi}

\newcommand{\BasisVec}{\boldsymbol \phi}
\newcommand{\BasisDiscrete}{\mathbf{\Phi}}
\newcommand{\BasisVecDiscrete}{\mathbf{\boldsymbol \phi}}
\newcommand{\NBasis}{K}
\newcommand{\ApproxPDEStateArgs}{\boldsymbol x, t,\params;\weights}
\newcommand{\GenState}{\hat{\boldsymbol u}}
\newcommand{\BasisArgs}{\boldsymbol x, t,\params;\weightsInner}
\newcommand{\BasisDiscreteArgs}{\params;\weightsInner}
\newcommand{\TrialSpace}{\mathcal{V}}
\newcommand{\velocity}{\boldsymbol f}
\newcommand{\ApproxSemiDiscreteState}{\tilde{\SemiDiscreteState}}
\newcommand{\SemiDiscreteStateRef}{\mathbf{u}_{\text{ref}}}
\newcommand{\SemiDiscreteSpatialTrialSpace}{\mathcal{V}}
\newcommand{\SemiDiscreteSpatialBasisMat}{\mathbf{V}}
\newcommand{\SemiDiscreteSpatialBasisVec}{\mathbf{v}}
\newcommand{\SemiDiscreteGenState}{\hat{\SemiDiscreteState}}
\newcommand{\SemiDiscreteGenStateArgs}{\SemiDiscreteStateArgs}


\newcommand{\FullyDiscreteStateDummy}{\mathbf{v}}

\newcommand{\FullyDiscreteState}{\mathbf{u}}
\newcommand{\FullyDiscreteStateArgs}{\params}

\newcommand{\residLMS}{\boldsymbol{r}}

\newcommand{\CollocationSet}{\mathcal{C}}
\newcommand{\NCollocation}{N_c}
\newcommand{\ResidualLoss}{\mathcal{L}_{r}}

\newcommand{\SemiDiscreteStateDum}{\boldsymbol v}

\newcommand{\SemiDiscreteState}{\boldsymbol u}
\newcommand{\SemiDiscreteStateArgs}{t,\params}
\newcommand{\SpaceTimeGenState}{\overline{\hat{\SemiDiscreteState}}}

\newcommand{\SemiDiscreteSpaceTimeGenState}{\overline{\hat{\SemiDiscreteState}}}
\newcommand{\ParametricSpaceTimeGenState}{\overline{\hat{\SemiDiscreteState}}}
\newcommand{\SemiDiscreteParametricSpaceTimeGenState}{\overline{\hat{\SemiDiscreteState}}}
\newcommand{\SemiDiscreteParametricSpaceTimeGenStateNN}{\overline{\hat{\boldsymbol z}}}
\newcommand{\ParametricSpaceTimeGenStateNN}{\overline{\hat{\boldsymbol z}}}

\newcommand{\DiscreteResid}{\mathbf{R}}
\newcommand{\DiscreteState}{\mathbf{u}}
\newcommand{\DiscreteStateArgs}{\params}
\newcommand{\numTimeSteps}{N_t}
\newcommand{\DiscreteSpaceTimeState}{\overline{\DiscreteState}}
\newcommand{\DiscreteSpaceTimeResid}{\overline{\DiscreteResid}}
\newcommand{\DiscreteSpaceTimeResidArgs}{\params}

\newcommand{\DiscreteResMin}{J}
\newcommand{\ContinuousResMin}{\mathcal{J}}
\newcommand{\QuadWeights}{\alpha}
\newcommand{\numQuadPointsParams}{N_p}
\newcommand{\activation}{\boldsymbol \eta}
\newcommand{\activationFunc}{\boldsymbol \zeta}

\newcommand{\Weights}{\mathbf{W}}
\newcommand{\biases}{\mathbf{b}}
\newcommand{\NumWeights}{N_{\weights}}

\begin{document}
\begin{frontmatter}

\title{Deep Bases: A practical deep learning approach for model reduction}
%\title{The windowed least-squares framework for model reduction of dynamical systems}

%\author[a]{Eric J. Parish and Kevin T. Carlberg}
%\ead{ejparis@sandia.gov}
\begin{abstract}
\end{abstract}
\end{frontmatter}


%\maketitle
\section{Introduction}
The simulation of parameterized partial differential equations is ubiquitous in computational science, playing important roles in uncertainty quantification, design and optimization, and control, to name a few areas. Despite the tremendous growth in computing capabilities in the past several decades, however, simulating partial differential equations for a single parameter instance can be a computational bottle neck. As a result, analysts often rely on low-cost computational models to generate approximate solutions. 

Projection-based reduced-order models comprise a promising approximation strategy. 

\begin{enumerate}
\item PROMs
\begin{itemize}
\item Project onto low-dimensional space-(time) subspace, predict for new parameter instances
\item Kolmogorov n-width
\end{itemize}
\item PINNs
\begin{itemize}
\item Learn a mapping via deep learning, minimize composite loss
\item Full-order model solve
\item Final layer is a basis
\end{itemize}
\end{enumerate}  
\section{Mathematical setting}
We consider model reduction of the partial differential equation given by
\begin{equation}\label{eq:fom}
\frac{\partial \PDEState}{\partial t}(\PDEStateArgs) = \PDEFlux(\PDEState,\PDEStateArgs), 
\end{equation}
where $\PDEState: \PDEStateDomain \rightarrow \PDEStateCodomain$ is the state, $\PDEFlux: \PDEStateDomain \times \paramDomain$ is the flux, and $\params \in \paramDomain \subset \RR{\numParams}$  are system parameters.
\subsection{Semi-discrete form}
In model reduction, it is often the case where we aim to generate reduced-order models not for the continuous PDE~\eqref{eq:fom}, but rather for a semi-discrete form obtained after spatial discretization. This is the case, for example, when working with legacy codes with complex equation sets, and when working with hyperbolic systems that require stabilization. To this end, we introduce a discretization of $\Omega$ into $\NSpace$ degrees of freedom characterized by the nodal points $\xd_i$, which in three-dimensions, for example, is given by the coordinates $\xd_i = (\xs_i,\ys_i,\zs_i)$. Spatial discrertization of the FOM PDE~\eqref{eq:fom} yields the semi-discrete system  
$$\frac{d \SemiDiscreteState}{dt}(\SemiDiscreteStateArgs) = \velocity(\SemiDiscreteState(\SemiDiscreteStateArgs),t,\params),$$
where $\SemiDiscreteState(t,\params) \in \RR{\FomDim}$ is the semi-discrete state-vector, $\FomDim = \NVars \NSpace$ the total of discrete degrees of freedom, and $\velocity: \RR{\FomDim} \times [0,T] \times \paramDomain \rightarrow \RR{\FomDim}$ is the velocity function.

 \subsection{Reduced-order models}
Projection-based ROMs generate approximate solutions to the FOM
	ODE~\eqref{eq:FOM} by approximating the state in a low-dimensional trial
	subspace. Two types of space--time trial subspaces are commonly used for
	this purpose:\footnote{For both spatial and space--time ROMs of dynamical systems, all trial subspaces are, strictly speaking, space--time subspaces.} 
\begin{enumerate} 
	\item \textit{Subspaces that reduce only the spatial dimension of the full-order
		model (\spatialAcronym)}. These trial subspaces are characterized by a spatial projection operator, associate with a basis that represents the spatial dependence of the state, and are employed in classic model reduction approaches, e.g., Galerkin and LSPG. %These 
	\item \textit{Subspaces that reduce both the spatial and temporal dimensions of the full-order
		model (\spaceTimeAcronym)}.
These trial subspaces are characterized by a space--time projection operator, associate with a basis that represents the spatial and temporal dependence of the state, and are employed in space--time 
model reduction approaches (e.g., space--time Galerkin~\cite{benner_st}, space--time LSPG~\cite{choi_stlspg}). 
\end{enumerate}
 We now describe these two types of space--time trial subspaces and their
	application to the Galerkin, LSPG, and space--time LSPG approaches. 

\subsection{\spatialAcronym\ trial subspaces}
At a given time instance 
$t\in[0,T]$ and parameter instance $\params \in \paramDomain$,
\spatialAcronym\ trial subspaces approximate the FOM ODE solution
	as $\ApproxSemiDiscreteState(\SemiDiscreteStateArgs)\approx\SemiDiscreteState(\SemiDiscreteStateArgs)$, which is enforced to reside in an
	affine spatial trial subspace of dimension $K\ll\FomDim$ such that
	$\ApproxSemiDiscreteState(\SemiDiscreteStateArgs) \in
	\SemiDiscreteStateRef +\SemiDiscreteSpatialTrialSpace
\subseteq\RR{\FomDim}$, where $\dim(\SemiDiscreteSpatialTrialSpace) = K$
and $\SemiDiscreteStateRef \in \mathbb{R}^{\FomDim}$ denotes the reference state, which
	is often taken to be the initial condition.
Here, the trial subspace
$\SemiDiscreteSpatialTrialSpace$ 
is spanned by an orthogonal basis such that
$ \SemiDiscreteSpatialTrialSpace= \Range{\SemiDiscreteSpatialBasisMat}$
with 
$ \SemiDiscreteSpatialBasisMat \equiv \begin{bmatrix}  \SemiDiscreteSpatialBasisVec_1  & \cdots &  \SemiDiscreteSpatialBasisVec_K \end{bmatrix}
	\in \RRStar{\RomDim}{\FomDim}$, where $\RRStar{\RomDim}{\FomDim}$ denotes the compact Stiefel manifold (i.e.,  $
	\RRStar{\RomDim}{\FomDim}\defeq
	\{ \mathbf{X} \in \RR{\FomDim
	\times \RomDim}\, \big|\, \mathbf{X}^T \mathbf{X} = \mathbf{I} \}$).
The basis vectors $\SemiDiscreteSpatialBasisVec_i$, $i=1,\ldots,K$ are typically constructed
using state snapshots, e.g., via
proper orthogonal decomposition (POD)~\cite{berkooz_turbulence_pod}, the reduced-basis method~\cite{rb_1,rb_2,rb_3,NgocCuong2005,Rozza2008}. 
Thus, at any time instance $t\in[0,T]$ and parameter instance $\params \in \paramDomain$, ROMs that employ the  \spatialAcronym\
trial subspace approximate the FOM ODE solution as
\begin{equation}\label{eq:affine_trialspace}
\SemiDiscreteState(\SemiDiscreteStateArgs)  \approx \ApproxSemiDiscreteState(\SemiDiscreteStateArgs) = \SemiDiscreteSpatialBasisMat  \SemiDiscreteGenState(\SemiDiscreteGenStateArgs) + \SemiDiscreteStateRef,
\end{equation}
where $\SemiDiscreteGenState(\SemiDiscreteGenStateArgs) \in \RR{\RomDim}$ denotes the generalized
coordinates. 
\begin{comment}

From the space--time perspective, this is equivalent to approximating the
	FOM ODE solution trajectory $\stateFOM\in\RR{N}\otimes\timeSpace$ with 
	$\approxstate\in \stspaceS$, where
\begin{equation}\label{eq:spatial_subspace}
\begin{split}
& \stspaceS \defeq \trialspace \otimes \timeSpace +
	\stateIntercept\otimes\onesFunction\subseteq\RR{N}\otimes\timeSpace,
\end{split}
\end{equation}
with $\onesFunction\in\timeSpace$ defined as
$\onesFunction:\timeDummy\mapsto 1$.
 
Substituting the approximation~\eqref{eq:affine_trialspace} into the FOM ODE~\eqref{eq:FOM} and performing orthogonal
$\elltwo$-projection of the initial condition onto the trial subspace yields
the overdetermined system of ODEs
\begin{equation}\label{eq:g_truncation}
\basisspace \genstateDotArg{}{t} = \velocity(\basisspace
\genstateArg{}{t} + \stateIntercept,t ), \qquad \genstate(0) = \genstateIC,
	\qquad t \in [0,T],
\end{equation}
where $\genstateDot\equiv {d \genstate}/{d\tau}$.
Because Eq.~\eqref{eq:g_truncation} is overdetermined, a solution may not
exist. Typically, either \textit{Galerkin} or \textit{least-squares
Petrov--Galerkin} projection is employed to reduce the number of equations
such that a unique solution exists. We now describe these two methods.

and the fully discrete system as
$$\DiscreteResid(\DiscreteState(\DiscreteStateArgs)) = \bz.$$ 
\end{comment}




\subsection{\spaceTimeAcronym\ trial spaces and space--time ROMs}
Space--time projection methods that employ \spaceTimeAcronym\ trial
spaces~\cite{choi_stlspg,constantine_strom,URBAN2012203,Yano2014ASC,benner_st,bui_thesis}
aim to overcome the latter two shortcomings of LSPG. Because these methods employ \spaceTimeAcronym\ trial
spaces, they reduce both the spatial and temporal dimensions of the full-order
model; further, they yield error bounds that grow more slowly in time and
their trajectories exhibit an optimality property over the entire time domain. 

For a given parameter instance $\params \in \paramDomain$, \spaceTimeAcronym\ trial subspaces approximate the FOM ODE solution
trajectory
	$\SemiDiscreteState(\cdot,\params) \in\RR{\FomDim}\otimes \TimeSpace$ with an approximation that resides in an
	affine space--time trial subspace of dimension $\STDim\ll\FomDim$, i.e., 
	$\ApproxSemiDiscreteState \in \SemiDiscreteSpaceTimeTrialSpace$ with $\dim(\SemiDiscreteSpaceTimeTrialSpace) =\STDim $, where
\begin{equation}\label{eq:sttrialspace_def}
 \SemiDiscreteSpaceTimeTrialSpace \defeq 
	\Range{\SemiDiscreteSpaceTimeBasisMat} + 
	\SemiDiscreteStateIC \otimes \OnesFunction
	\subseteq \RR{\FomDim} \otimes \TimeSpace.
\end{equation}

%and $\stbasis\in\RR{N \times \stdim}\otimes \timeSpace$ with 
%Here $\stbasis \in \RR{\fomdim \times \stdim} \otimes \timeSpace$, with $\stbasis:\timeDummy\mapsto\stbasis(\timeDummy)$ and $\stbasis(0) =
%\boldsymbol 0$
%denotes the space--time trial basis. 
%Thus, at any time instance $t\in[0,T]$, ROMs that employ the
%\spaceTimeAcronym\ trial subspace approximate the FOM ODE solution as
%\begin{equation}\label{eq:stapprox1}
% \stateFOMArg{}{t} \approx \approxstateArg{}{t}  = \stbasisArg{}{t} \stgenstate + \stateFOMIC,
%\end{equation}
%where $\stgenstateArg{} \in \RR{\stdim}$ denotes the space--time generalized coordinates. 
%Critically, comparing the approximations arising from \spatialAcronym\ and
%\spaceTimeAcronym\ trial subspaces in Eqs.~\eqref{eq:affine_trialspace} and \eqref{eq:stapprox1}, respectively,
%highlights that the former approximation associates with time-dependent
%generalized coordinates, while the latter approximation associates with a
%time-dependent basis matrix.











\section{Deep learning surrogate for the state}
In the offline phase, we model our state as
$$\PDEState(\PDEStateArgs) \approx \ApproxPDEState(\ApproxPDEStateArgs),$$
where $\ApproxPDEState$ is a deep learning network whose last layer is \textbf{linear} with no bias. Denoting this final layer as $\Basis$, we can write our approximation as
$$\ApproxPDEState(\ApproxPDEStateArgs) = \Basis (\BasisArgs)\GenState \equiv \sum_{i=1}^{\NBasis} \BasisVec_i(\BasisArgs) \GenState_i.$$
Given a space--time grid, we can define a space--time discrete basis vector as
$$\BasisVecDiscrete_i(\BasisDiscreteArgs) = \begin{bmatrix} 
\BasisVec_i(\x_1,t_1,\params;\weightsInner) & \cdots & \BasisVec_i(\x_{\NST},t_{\NST},\params;\weightsInner)
\end{bmatrix}^T$$
and a discrete space--time basis matrix as
$$\BasisDiscrete(\BasisDiscreteArgs) = \begin{bmatrix} \BasisVecDiscrete_1(\BasisDiscreteArgs) & \cdots & \BasisVecDiscrete_{\NBasis}(\BasisDiscreteArgs) \end{bmatrix}.$$
\section{Discrete formulation, and discrete space--time least-squares Petrov--Galerkin}
We consider the discretization of the state at time instances $n=1,\ldots,\numTimeSteps$,
$$\DiscreteSpaceTimeState = \begin{bmatrix} [\DiscreteState^1]^T & \cdots & [\DiscreteState^{\numTimeSteps}]^T \end{bmatrix}^T.$$
Analogously, we define the space--time residual as
$$\DiscreteSpaceTimeResid: \DiscreteSpaceTimeState \mapsto \DiscreteSpaceTimeResid(\DiscreteSpaceTimeState;\DiscreteSpaceTimeResidArgs).$$i
\subsection{Space--time Galerkin}
The space--time Galerkin method operates by restricting the state to the trial space, and enforces the residual to be orthogonal to the trial subspace,


\section{Galerkin reduced-order model}
The Galerkin ROM operates by restricting the state to be orthogonal to the trial space. At a continuous level, for a given parameter instance, we solve the following weak problem: find $\ApproxPDEState \in \TrialSpace$ such that
$$\left( \BasisVec, \PDEFlux\left(\ApproxPDEState\left(\PDEStateArgs\right)\right) \right) = \bz \qquad \forall \BasisVec \in \TrialSpace$$
where $\TrialSpace \equiv \text{span}\{ \BasisVec_i\}_{i=1}^{\NBasis}$.

\end{document}

