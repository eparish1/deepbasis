%\documentclass{article}
\documentclass[3p,computermodern,10pt]{elsarticle}
\input{packages}
\newcommand{\timeDummy}{\tau}
\newcommand{\spatialAcronym}{SDR}
\newcommand{\spaceTimeAcronym}{STDR}
\newcommand{\parametricSpaceTimeAcronym}{PSTDR}
\newcommand{\MLSubspaceNameLowercase}{deep subspaces}
%\newcommand{\span}[1]{\text{span}{#1}}
\newcommand{\PSTWeightingMatrix}{\boldsymbol W}
\newcommand{\PSTInnerProduct}[2]{\left( #1, #2 \right)_{*}}
\newcommand{\defeq}{\vcentcolon=}
\newcommand{\RRStar}[2]{\mathbb{V}_{#1}(\RR{#2})}
\newcommand{\Range}[1]{\text{Range}(#1)}
\newcommand{\RomDim}{K}
\newcommand{\NVars}{N_v}
\newcommand{\ParametricSpace}{\mathcal{P}}
\newcommand{\TimeDomain}{[0,T]}
\newcommand{\paramDomain}{\mathcal{D}}
\newcommand{\numParams}{N_{\mu}}
\newcommand{\PhysicalDomain}{\Omega}
\newcommand{\PDEStateDomain}{\PhysicalDomain \times (0,T] \times \paramDomain}
\newcommand{\PDEStateCodomain}{\RR{\NVars}}
\newcommand{\PDEFlux}{\mathcal{F}}
\newcommand{\NSpace}{N_s}
\newcommand{\NTime}{\numTimeSteps}
\newcommand{\NST}{N_{st}}
\newcommand{\x}{\boldsymbol x}
\newcommand{\xd}{\mathbf{x}}
\newcommand{\xs}{x}
\newcommand{\ys}{y}
\newcommand{\zs}{z}
\newcommand{\NeuralNetwork}{\mathcal{NN}}
\newcommand{\TimeSpace}{\mathcal{T}}
\newcommand{\STDim}{K_{\text{st}}}
\newcommand{\PSTDim}{K_{\text{pst}}}

\newcommand{\SemiDiscreteSpaceTimeTrialSpace}{\mathcal{ST}}
\newcommand{\SemiDiscreteParametricSpaceTimeTrialSpace}{\mathcal{PST}}

\newcommand{\SemiDiscreteSpaceTimeBasisMat}{\boldsymbol \Pi}
\newcommand{\OnesFunction}{\mathcal{O}}
\newcommand{\SemiDiscreteParametricSpaceTimeBasisMat}{\boldsymbol \Phi}
\newcommand{\SemiDiscreteParametricSpaceTimeBasisVec}{\boldsymbol \phi}

\newcommand{\TrainingData}{\mathcal{S}}
\newcommand{\NTrain}{N_{\text{train}}}
\newcommand{\SemiDiscreteStateIC}{\SemiDiscreteState_0(\params)}
\newcommand{\SemiDiscreteStateParametricSpaceTimeRef}{\SemiDiscreteState_{\text{ref}}}

\newcommand{\RR}[1]{\mathbb{R}^{#1}}
\newcommand{\FomDim}{N}
\newcommand{\PDEState}{\mathsf{u}}
\newcommand{\ApproxPDEState}{\tilde{\boldsymbol u}}
\newcommand{\NNWeights}{\boldsymbol \theta}
\newcommand{\NWeights}{N_{\NNWeights}}
\newcommand{\weightsInner}{\boldsymbol \theta^*}
\newcommand{\bz}{\boldsymbol 0}
\newcommand{\params}{\boldsymbol \mu}
\newcommand{\PDEStateArgs}{\x, t,\params}
\newcommand{\weights}{\boldsymbol \theta}
\newcommand{\Basis}{\boldsymbol \Phi}
\newcommand{\BasisVec}{\boldsymbol \phi}
\newcommand{\BasisDiscrete}{\mathbf{\Phi}}
\newcommand{\BasisVecDiscrete}{\mathbf{\boldsymbol \phi}}
\newcommand{\NBasis}{K}
\newcommand{\ApproxPDEStateArgs}{\boldsymbol x, t,\params;\weights}
\newcommand{\GenState}{\hat{\boldsymbol u}}
\newcommand{\BasisArgs}{\boldsymbol x, t,\params;\weightsInner}
\newcommand{\BasisDiscreteArgs}{\params;\weightsInner}
\newcommand{\TrialSpace}{\mathcal{V}}
\newcommand{\velocity}{\boldsymbol f}
\newcommand{\ApproxSemiDiscreteState}{\tilde{\SemiDiscreteState}}
\newcommand{\SemiDiscreteStateRef}{\SemiDiscreteState_{\text{ref}}}
\newcommand{\SemiDiscreteSpatialTrialSpace}{\mathcal{V}}
\newcommand{\SemiDiscreteSpatialBasisMat}{\mathbf{V}}
\newcommand{\SemiDiscreteSpatialBasisVec}{\mathbf{v}}
\newcommand{\SemiDiscreteGenState}{\hat{\SemiDiscreteState}}
\newcommand{\SemiDiscreteGenStateArgs}{\SemiDiscreteStateArgs}


\newcommand{\SemiDiscreteState}{\boldsymbol u}
\newcommand{\SemiDiscreteStateArgs}{t,\params}
\newcommand{\SemiDiscreteSpaceTimeGenState}{\overline{\hat{\SemiDiscreteState}}}
\newcommand{\SemiDiscreteParametricSpaceTimeGenState}{\overline{\hat{\SemiDiscreteState}}}
\newcommand{\SemiDiscreteParametricSpaceTimeGenStateNN}{\overline{\hat{\boldsymbol z}}}

\newcommand{\DiscreteResid}{\mathbf{R}}
\newcommand{\DiscreteState}{\mathbf{u}}
\newcommand{\DiscreteStateArgs}{\params}
\newcommand{\numTimeSteps}{N_t}
\newcommand{\DiscreteSpaceTimeState}{\overline{\DiscreteState}}
\newcommand{\DiscreteSpaceTimeResid}{\overline{\DiscreteResid}}
\newcommand{\DiscreteSpaceTimeResidArgs}{\params}

\begin{document}
\begin{frontmatter}

\title{Deep Bases: A practical deep learning approach for model reduction}
%\title{The windowed least-squares framework for model reduction of dynamical systems}

%\author[a]{Eric J. Parish and Kevin T. Carlberg}
%\ead{ejparis@sandia.gov}
\begin{abstract}
\end{abstract}
\end{frontmatter}


%\maketitle
\section{Introduction}
We consider model reduction of the partial differential equation given by
\begin{equation}\label{eq:fom}
\PDEResid(\PDEState(\PDEStateArgs)) = \bz.
\end{equation}
\subsection{Semi-discrete form}
We now discuss a form of Eq.~\eqref{eq:fom} in a semi-discrete form. To this end, we introduce a discretization of $\Omega$ into $N$ 
Typically, in model reduction, we either work directly with the continuous form~\eqref{eq:fom}, a semi-discrete form of~\eqref{eq:fom} (obtained, for example, by discretizing in space), or a fully-discrete form of the system. We denote the semi-discrete system as 
$$\frac{d \state}{dt} = \velocity(\state,t,\params),$$
where $\state(t) \in \RR{\FomDim}$ is the disc
and the fully discrete system as
$$\DiscreteResid(\DiscreteState(\DiscreteStateArgs)) = \bz.$$ 

\section{Deep learning surrogate for the state}
In the offline phase, we model our state as
$$\PDEState(\PDEStateArgs) \approx \ApproxPDEState(\ApproxPDEStateArgs),$$
where $\ApproxPDEState$ is a deep learning network whose last layer is \textbf{linear} with no bias. Denoting this final layer as $\Basis$, we can write our approximation as
$$\ApproxPDEState(\ApproxPDEStateArgs) = \Basis (\BasisArgs)\GenState \equiv \sum_{i=1}^{\NBasis} \BasisVec_i(\BasisArgs) \GenState_i.$$
Given a space--time grid, we can define a space--time discrete basis vector as
$$\BasisVecDiscrete_i(\BasisDiscreteArgs) = \begin{bmatrix} 
\BasisVec_i(\x_1,t_1,\params;\weightsInner) & \cdots & \BasisVec_i(\x_{\NST},t_{\NST},\params;\weightsInner)
\end{bmatrix}^T$$
and a discrete space--time basis matrix as
$$\BasisDiscrete(\BasisDiscreteArgs) = \begin{bmatrix} \BasisVecDiscrete_1(\BasisDiscreteArgs) & \cdots & \BasisVecDiscrete_{\NBasis}(\BasisDiscreteArgs) \end{bmatrix}.$$
\section{Discrete formulation, and discrete space--time least-squares Petrov--Galerkin}
We consider the discretization of the state at time instances $n=1,\ldots,\numTimeSteps$,
$$\DiscreteSpaceTimeState = \begin{bmatrix} [\DiscreteState^1]^T & \cdots & [\DiscreteState^{\numTimeSteps}]^T \end{bmatrix}^T.$$
Analogously, we define the space--time residual as
$$\DiscreteSpaceTimeResid: \DiscreteSpaceTimeState \mapsto \DiscreteSpaceTimeResid(\DiscreteSpaceTimeState;\DiscreteSpaceTimeResidArgs).$$i
\subsection{Space--time Galerkin}
The space--time Galerkin method operates by restricting the state to the trial space, and enforces the residual to be orthogonal to the trial subspace,


\section{Galerkin reduced-order model}
The Galerkin ROM operates by restricting the state to be orthogonal to the trial space. At a continuous level, for a given parameter instance, we solve the following weak problem: find $\ApproxPDEState \in \TrialSpace$ such that
$$\left( \BasisVec, \PDEResid\left(\ApproxPDEState\left(\PDEStateArgs\right)\right) \right) = \bz \qquad \forall \BasisVec \in \TrialSpace$$
where $\TrialSpace \equiv \text{span}\{ \BasisVec_i\}_{i=1}^{\NBasis}$.

\end{document}

